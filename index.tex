% Options for packages loaded elsewhere
\PassOptionsToPackage{unicode}{hyperref}
\PassOptionsToPackage{hyphens}{url}
\PassOptionsToPackage{dvipsnames,svgnames,x11names}{xcolor}
%
\documentclass[
  letterpaper,
  DIV=11,
  numbers=noendperiod]{scrreprt}

\usepackage{amsmath,amssymb}
\usepackage{lmodern}
\usepackage{iftex}
\ifPDFTeX
  \usepackage[T1]{fontenc}
  \usepackage[utf8]{inputenc}
  \usepackage{textcomp} % provide euro and other symbols
\else % if luatex or xetex
  \usepackage{unicode-math}
  \defaultfontfeatures{Scale=MatchLowercase}
  \defaultfontfeatures[\rmfamily]{Ligatures=TeX,Scale=1}
\fi
% Use upquote if available, for straight quotes in verbatim environments
\IfFileExists{upquote.sty}{\usepackage{upquote}}{}
\IfFileExists{microtype.sty}{% use microtype if available
  \usepackage[]{microtype}
  \UseMicrotypeSet[protrusion]{basicmath} % disable protrusion for tt fonts
}{}
\makeatletter
\@ifundefined{KOMAClassName}{% if non-KOMA class
  \IfFileExists{parskip.sty}{%
    \usepackage{parskip}
  }{% else
    \setlength{\parindent}{0pt}
    \setlength{\parskip}{6pt plus 2pt minus 1pt}}
}{% if KOMA class
  \KOMAoptions{parskip=half}}
\makeatother
\usepackage{xcolor}
\setlength{\emergencystretch}{3em} % prevent overfull lines
\setcounter{secnumdepth}{5}
% Make \paragraph and \subparagraph free-standing
\ifx\paragraph\undefined\else
  \let\oldparagraph\paragraph
  \renewcommand{\paragraph}[1]{\oldparagraph{#1}\mbox{}}
\fi
\ifx\subparagraph\undefined\else
  \let\oldsubparagraph\subparagraph
  \renewcommand{\subparagraph}[1]{\oldsubparagraph{#1}\mbox{}}
\fi


\providecommand{\tightlist}{%
  \setlength{\itemsep}{0pt}\setlength{\parskip}{0pt}}\usepackage{longtable,booktabs,array}
\usepackage{calc} % for calculating minipage widths
% Correct order of tables after \paragraph or \subparagraph
\usepackage{etoolbox}
\makeatletter
\patchcmd\longtable{\par}{\if@noskipsec\mbox{}\fi\par}{}{}
\makeatother
% Allow footnotes in longtable head/foot
\IfFileExists{footnotehyper.sty}{\usepackage{footnotehyper}}{\usepackage{footnote}}
\makesavenoteenv{longtable}
\usepackage{graphicx}
\makeatletter
\def\maxwidth{\ifdim\Gin@nat@width>\linewidth\linewidth\else\Gin@nat@width\fi}
\def\maxheight{\ifdim\Gin@nat@height>\textheight\textheight\else\Gin@nat@height\fi}
\makeatother
% Scale images if necessary, so that they will not overflow the page
% margins by default, and it is still possible to overwrite the defaults
% using explicit options in \includegraphics[width, height, ...]{}
\setkeys{Gin}{width=\maxwidth,height=\maxheight,keepaspectratio}
% Set default figure placement to htbp
\makeatletter
\def\fps@figure{htbp}
\makeatother

\newcommand{\bs}{\symbf}
\newcommand{\mb}{\symbf}
\newcommand{\E}{\mathbb{E}}
\newcommand{\V}{\mathbb{V}}
\newcommand{\var}{\text{var}}
\newcommand{\cov}{\text{cov}}
\newcommand{\N}{\mathcal{N}}
\newcommand{\Bern}{\text{Bern}}
\newcommand{\Bin}{\text{Bin}}
\newcommand{\Pois}{\text{Pois}}
\newcommand{\Unif}{\text{Unif}}
\newcommand{\se}{\textsf{se}}
\newcommand{\au}{\underline{a}}
\newcommand{\du}{\underline{d}}
\newcommand{\Au}{\underline{A}}
\newcommand{\Du}{\underline{D}}
\newcommand{\xu}{\underline{x}}
\newcommand{\Xu}{\underline{X}}
\newcommand{\Yu}{\underline{Y}}
\renewcommand{\P}{\mathbb{P}}
\newcommand{\U}{\mb{U}}
\newcommand{\Xbar}{\overline{X}}
\newcommand{\Ybar}{\overline{Y}}
\newcommand{\real}{\mathbb{R}}
\newcommand{\bbL}{\mathbb{L}}
\renewcommand{\u}{\mb{u}}
\renewcommand{\v}{\mb{v}}
\newcommand{\M}{\mb{M}}
\newcommand{\X}{\mb{X}}
\newcommand{\Xmat}{\mathbb{X}}
\newcommand{\bfx}{\mb{x}}
\newcommand{\y}{\mb{y}}
\newcommand{\bfbeta}{\mb{\beta}}
\renewcommand{\b}{\symbf{\beta}}
\newcommand{\e}{\bs{\epsilon}}
\newcommand{\bhat}{\widehat{\mb{\beta}}}
\newcommand{\XX}{\Xmat'\Xmat}
\newcommand{\XXinv}{\left(\XX\right)^{-1}}
\newcommand{\hatsig}{\hat{\sigma}^2}
\newcommand{\red}[1]{\textcolor{red!60}{#1}}
\newcommand{\indianred}[1]{\textcolor{indianred}{#1}}
\newcommand{\blue}[1]{\textcolor{blue!60}{#1}}
\newcommand{\dblue}[1]{\textcolor{dodgerblue}{#1}}
\newcommand{\indep}{\perp\!\!\!\perp}
\newcommand{\inprob}{\overset{p}{\to}}
\newcommand{\indist}{\overset{d}{\to}}
\newcommand{\eframe}{\end{frame}}
\newcommand{\bframe}{\begin{frame}}
\newcommand{\R}{\textsf{\textbf{R}}}
\newcommand{\Rst}{\textsf{\textbf{RStudio}}}
\newcommand{\rfun}[1]{\texttt{\color{magenta}{#1}}}
\newcommand{\rpack}[1]{\textbf{#1}}
\newcommand{\rexpr}[1]{\texttt{\color{magenta}{#1}}}
\newcommand{\filename}[1]{\texttt{\color{blue}{#1}}}
\DeclareMathOperator*{\argmax}{arg\,max}
\DeclareMathOperator*{\argmin}{arg\,min}
\KOMAoption{captions}{tableheading}
\makeatletter
\@ifpackageloaded{tcolorbox}{}{\usepackage[many]{tcolorbox}}
\@ifpackageloaded{fontawesome5}{}{\usepackage{fontawesome5}}
\definecolor{quarto-callout-color}{HTML}{909090}
\definecolor{quarto-callout-note-color}{HTML}{0758E5}
\definecolor{quarto-callout-important-color}{HTML}{CC1914}
\definecolor{quarto-callout-warning-color}{HTML}{EB9113}
\definecolor{quarto-callout-tip-color}{HTML}{00A047}
\definecolor{quarto-callout-caution-color}{HTML}{FC5300}
\definecolor{quarto-callout-color-frame}{HTML}{acacac}
\definecolor{quarto-callout-note-color-frame}{HTML}{4582ec}
\definecolor{quarto-callout-important-color-frame}{HTML}{d9534f}
\definecolor{quarto-callout-warning-color-frame}{HTML}{f0ad4e}
\definecolor{quarto-callout-tip-color-frame}{HTML}{02b875}
\definecolor{quarto-callout-caution-color-frame}{HTML}{fd7e14}
\makeatother
\makeatletter
\makeatother
\makeatletter
\@ifpackageloaded{bookmark}{}{\usepackage{bookmark}}
\makeatother
\makeatletter
\@ifpackageloaded{caption}{}{\usepackage{caption}}
\AtBeginDocument{%
\ifdefined\contentsname
  \renewcommand*\contentsname{Table of contents}
\else
  \newcommand\contentsname{Table of contents}
\fi
\ifdefined\listfigurename
  \renewcommand*\listfigurename{List of Figures}
\else
  \newcommand\listfigurename{List of Figures}
\fi
\ifdefined\listtablename
  \renewcommand*\listtablename{List of Tables}
\else
  \newcommand\listtablename{List of Tables}
\fi
\ifdefined\figurename
  \renewcommand*\figurename{Figure}
\else
  \newcommand\figurename{Figure}
\fi
\ifdefined\tablename
  \renewcommand*\tablename{Table}
\else
  \newcommand\tablename{Table}
\fi
}
\@ifpackageloaded{float}{}{\usepackage{float}}
\floatstyle{ruled}
\@ifundefined{c@chapter}{\newfloat{codelisting}{h}{lop}}{\newfloat{codelisting}{h}{lop}[chapter]}
\floatname{codelisting}{Listing}
\newcommand*\listoflistings{\listof{codelisting}{List of Listings}}
\usepackage{amsthm}
\theoremstyle{definition}
\newtheorem{definition}{Definition}[chapter]
\theoremstyle{remark}
\renewcommand*{\proofname}{Proof}
\newtheorem*{remark}{Remark}
\newtheorem*{solution}{Solution}
\makeatother
\makeatletter
\@ifpackageloaded{caption}{}{\usepackage{caption}}
\@ifpackageloaded{subcaption}{}{\usepackage{subcaption}}
\makeatother
\makeatletter
\@ifpackageloaded{tcolorbox}{}{\usepackage[many]{tcolorbox}}
\makeatother
\makeatletter
\@ifundefined{shadecolor}{\definecolor{shadecolor}{rgb}{.97, .97, .97}}
\makeatother
\makeatletter
\makeatother
\ifLuaTeX
  \usepackage{selnolig}  % disable illegal ligatures
\fi
\IfFileExists{bookmark.sty}{\usepackage{bookmark}}{\usepackage{hyperref}}
\IfFileExists{xurl.sty}{\usepackage{xurl}}{} % add URL line breaks if available
\urlstyle{same} % disable monospaced font for URLs
\hypersetup{
  pdftitle={STAT 210 Section Notes},
  pdfauthor={Zad Chin \& Jarell Cheong},
  colorlinks=true,
  linkcolor={blue},
  filecolor={Maroon},
  citecolor={Blue},
  urlcolor={Blue},
  pdfcreator={LaTeX via pandoc}}

\title{STAT 210 Section Notes}
\author{Zad Chin \& Jarell Cheong}
\date{}

\begin{document}
\maketitle
\ifdefined\Shaded\renewenvironment{Shaded}{\begin{tcolorbox}[interior hidden, boxrule=0pt, sharp corners, borderline west={3pt}{0pt}{shadecolor}, enhanced, breakable, frame hidden]}{\end{tcolorbox}}\fi

\renewcommand*\contentsname{Table of contents}
{
\hypersetup{linkcolor=}
\setcounter{tocdepth}{2}
\tableofcontents
}
\bookmarksetup{startatroot}

\hypertarget{preface}{%
\chapter*{Preface}\label{preface}}
\addcontentsline{toc}{chapter}{Preface}

\markboth{Preface}{Preface}

\includegraphics{./assets/img/stat210-logo.png}

This is a set of section notes for STAT 210: Probability I, a graduate
level probability course, at Harvard University taught by Professor
\href{mailto:blitz@g.harvard.edu}{Joe Blitzstein}, created by
\href{mailto:zadchin@college.harvard.edu}{Zad Chin} \&
\href{mailto:jarellcheong@college.harvard.edu}{Jarell Cheong Tze Wen}.

\hypertarget{important-logistics}{%
\section*{Important Logistics}\label{important-logistics}}
\addcontentsline{toc}{section}{Important Logistics}

\markright{Important Logistics}

\begin{itemize}
\tightlist
\item
  Section Time: 12:00 - 1:00 pm every Friday at SC316
\item
  Office Hour: 1:15 - 3:15pm every Friday at SC316
\end{itemize}

In Section, we will only briefly go through definition highlights in
class, and focus more on pencil problems that will strengthen
understandings for those concepts/definitions. Section note will be
posted here before Friday (latest by Thursday night), and section note
with solution will be posted after section (on Friday night).

In office hour, we will mainly discuss concept or problem on this week's
homework. Besides Section \& Office Hours, please feel free to ask
questions on Ed, or email Joe or us as well.

\hypertarget{notes}{%
\section*{Notes}\label{notes}}
\addcontentsline{toc}{section}{Notes}

\markright{Notes}

The section notes will be presented as a book, and the source for this
book at \url{https://github.com/zadchin/STAT210_Section}. Any typos or
errors can be reported at
\url{https://github.com/zadchin/STAT210_Section/issues}. Thanks for
reading.

This is a Quarto book. To learn more about Quarto books visit
\url{https://quarto.org/docs/books}.

\hypertarget{acknowledgement}{%
\section*{Acknowledgement}\label{acknowledgement}}
\addcontentsline{toc}{section}{Acknowledgement}

\markright{Acknowledgement}

We extend our profound gratitude to the Department of Statistics at
Harvard University, with a special acknowledgment to Professor Joe
Blitzstein for his invaluable insights and knowledge-sharing. Our
appreciation also goes to our peers from STAT 210, whose contributions
to discussions on the textbook problems have been invaluable. Finally,
we are indebted to the Stat-exchange platform for facilitating
discussions that addressed challenges we encountered, which have
subsequently enriched the content of our section notes.

\(\,\)

\bookmarksetup{startatroot}

\hypertarget{section-1}{%
\chapter*{Section 1}\label{section-1}}
\addcontentsline{toc}{chapter}{Section 1}

\markboth{Section 1}{Section 1}

Date: 15 Sept 2023

\hypertarget{introduction}{%
\section*{Introduction}\label{introduction}}
\addcontentsline{toc}{section}{Introduction}

\markright{Introduction}

In this section, we will discuss:

\begin{itemize}
\tightlist
\item
  \(\sigma\)-algebra
\item
  Borel \(\sigma\)-algebra
\item
  Probability measure
\item
  Random Variables \& Random Vectors
\item
  Limits of Events
\end{itemize}

\hypertarget{sigma-algebras}{%
\section*{\texorpdfstring{\(\sigma\)-algebras}{\textbackslash sigma-algebras}}\label{sigma-algebras}}
\addcontentsline{toc}{section}{\(\sigma\)-algebras}

\markright{\(\sigma\)-algebras}

In STAT 210 textbook, we define \(\sigma\)-algebra as follow:

\emph{Definition 2.2.1 from textbook}

\leavevmode\vadjust pre{\hypertarget{def-sigma-algebra}{}}%
\begin{definition}[]\label{def-sigma-algebra}

A \(\sigma\)-algebra on \(\Omega\) is a collection \(\mathcal{F}\) of
subsets of \(\Omega\) such that

\begin{enumerate}
\def\labelenumi{\arabic{enumi}.}
\item
  \(\emptyset \in \mathcal{F}\)
\item
  If \(A \in \mathcal{F}\), then \(A^c \in \mathcal{F}\)
\item
  If \(A_1, A_2, \cdots \in \mathcal{F}\), then
  \(\cup_{j=1}^{\infty} A_j \in \mathcal{F}\)
\end{enumerate}

\end{definition}

\hypertarget{pencil-2.2.5}{%
\subsection*{✏️ Pencil 2.2.5}\label{pencil-2.2.5}}
\addcontentsline{toc}{subsection}{✏️ Pencil 2.2.5}

Show that \(\sigma\)-algebra is automatically closed under countable
intersections, i.e., if \(A_1, A_2, \cdots\) are in \(\sigma\)-algebra
\(\mathcal{F}\), then \(\cap_{j=1}^{\infty} A_j \in \mathcal{F}\). Then
show that the requirement \(\emptyset \in \mathcal{F}\) can be replaced
by the requirements \(\mathcal{F} \neq \emptyset\).

\begin{tcolorbox}[enhanced jigsaw, arc=.35mm, colback=white, leftrule=.75mm, rightrule=.15mm, toprule=.15mm, bottomrule=.15mm, breakable, opacityback=0, left=2mm, colframe=quarto-callout-tip-color-frame]
\begin{minipage}[t]{5.5mm}
\textcolor{quarto-callout-tip-color}{\faLightbulb}
\end{minipage}%
\begin{minipage}[t]{\textwidth - 5.5mm}

\textbf{Solution}\vspace{2mm}

Will be added after section

\end{minipage}%
\end{tcolorbox}

\begin{tcolorbox}[enhanced jigsaw, colbacktitle=quarto-callout-note-color!10!white, leftrule=.75mm, coltitle=black, rightrule=.15mm, toptitle=1mm, bottomrule=.15mm, breakable, arc=.35mm, colback=white, titlerule=0mm, bottomtitle=1mm, toprule=.15mm, title=\textcolor{quarto-callout-note-color}{\faInfo}\hspace{0.5em}{Tip: De Morgan Laws}, left=2mm, opacityback=0, opacitybacktitle=0.6, colframe=quarto-callout-note-color-frame]

\[\left(\bigcup_{n=1}^{\infty} A_n \right)^c = \bigcap_{n=1}^{\infty} A_n ^c\]

\[\left(\bigcap_{n=1}^{\infty} A_n \right)^c = \bigcup_{n=1}^{\infty} A_n ^c\]

\end{tcolorbox}

\hypertarget{pencil-2.2.6}{%
\subsection*{✏️ Pencil 2.2.6}\label{pencil-2.2.6}}
\addcontentsline{toc}{subsection}{✏️ Pencil 2.2.6}

Check that an intersection of \(\sigma-\)algebras, even uncountably
many, is \(\sigma\)-algebra. Give a simple example showing that a union
of \(\sigma\)-algebras may not be a \(\sigma\)-algebra.

\begin{tcolorbox}[enhanced jigsaw, arc=.35mm, colback=white, leftrule=.75mm, rightrule=.15mm, toprule=.15mm, bottomrule=.15mm, breakable, opacityback=0, left=2mm, colframe=quarto-callout-tip-color-frame]
\begin{minipage}[t]{5.5mm}
\textcolor{quarto-callout-tip-color}{\faLightbulb}
\end{minipage}%
\begin{minipage}[t]{\textwidth - 5.5mm}

\textbf{Solution}\vspace{2mm}

Will be added after section

\end{minipage}%
\end{tcolorbox}

\begin{tcolorbox}[enhanced jigsaw, arc=.35mm, colback=white, leftrule=.75mm, rightrule=.15mm, toprule=.15mm, bottomrule=.15mm, breakable, opacityback=0, left=2mm, colframe=quarto-callout-important-color-frame]
\begin{minipage}[t]{5.5mm}
\textcolor{quarto-callout-important-color}{\faExclamation}
\end{minipage}%
\begin{minipage}[t]{\textwidth - 5.5mm}

As noted in the textbook as well as hazard 2.2.2: Note that a
\(\sigma\)-algebra \(\mathcal{F}\) is a collection of sets, and an
element of \(\mathcal{F}\) is a subset of \(\omega\). An interesection
of \(\sigma\)-algebras is very different from an intersection of events!

\end{minipage}%
\end{tcolorbox}

\hypertarget{borel-sigma-algebra}{%
\section*{\texorpdfstring{Borel
\(\sigma\)-algebra}{Borel \textbackslash sigma-algebra}}\label{borel-sigma-algebra}}
\addcontentsline{toc}{section}{Borel \(\sigma\)-algebra}

\markright{Borel \(\sigma\)-algebra}

\emph{Definition 2.3.1 from textbook}

\leavevmode\vadjust pre{\hypertarget{def-borel-sigma-algebra}{}}%
\begin{definition}[]\label{def-borel-sigma-algebra}

The Borel \(\sigma\)-algebra \(\mathcal{B}\) on \(\mathbb{R}\) is
definied to be the \(\sigma\)-algebra generated by all open intervals
\((a,b)\) with \(a,b\in \mathbb{R}\). A \emph{Borel Set} is a set in the
Borel \(\sigma\)-algebra.

\end{definition}

\hypertarget{pencil-2.3.2}{%
\subsection*{✏️ Pencil 2.3.2}\label{pencil-2.3.2}}
\addcontentsline{toc}{subsection}{✏️ Pencil 2.3.2}

Show that the Borel \(\sigma\)-algebra is the same if we use closed
intervals \([a,b]\) instead of open intervals to generate it, and that
it is also the same if we use ``semi-inifinite'' intervals
\((-\infty,b)\) to generate it.

\begin{tcolorbox}[enhanced jigsaw, arc=.35mm, colback=white, leftrule=.75mm, rightrule=.15mm, toprule=.15mm, bottomrule=.15mm, breakable, opacityback=0, left=2mm, colframe=quarto-callout-tip-color-frame]
\begin{minipage}[t]{5.5mm}
\textcolor{quarto-callout-tip-color}{\faLightbulb}
\end{minipage}%
\begin{minipage}[t]{\textwidth - 5.5mm}

\textbf{Solution}\vspace{2mm}

Will be added after section

\end{minipage}%
\end{tcolorbox}

\hypertarget{probability-measure}{%
\section*{Probability Measure}\label{probability-measure}}
\addcontentsline{toc}{section}{Probability Measure}

\markright{Probability Measure}

\emph{Section 2.5 from textbook}

\leavevmode\vadjust pre{\hypertarget{def-probability}{}}%
\begin{definition}[]\label{def-probability}

A \emph{probability space} \((\Omega, \mathcal{F}, P)\) consists of a
measurable space \((\Omega, \mathcal{F})\) and a map (the probability
function) \(P\) satisfying the following axioms:

\begin{enumerate}
\def\labelenumi{\arabic{enumi}.}
\tightlist
\item
  \(P(\emptyset) = 0\), \(P(\Omega) = 1\)
\item
  \(\forall A \in \mathcal{F}\), \(P(A) \geq 0\)
\item
  If \(\{A_i\}^{\infty}_{i=1}\) is a collection of mutually disjoint
  sets in \(\mathcal{F}\), i.e.~\(\forall i \neq j\),
  \(A_i \cap A_j = \emptyset\), then
  \[P\left( \bigcup_{i=1}^{\infty} A_i \right) = \sum_{i=1}^{\infty} P(A_i) \]
  \textbackslash end\{itemize\} This property is known as
  \(\sigma\)-additivity.
\end{enumerate}

\end{definition}

\hypertarget{pencil-2.5.1}{%
\subsection*{✏️ Pencil 2.5.1}\label{pencil-2.5.1}}
\addcontentsline{toc}{subsection}{✏️ Pencil 2.5.1}

Show that in the first axiom, the condition \(P(\emptyset) = 0\) can be
eliminated.

\begin{tcolorbox}[enhanced jigsaw, arc=.35mm, colback=white, leftrule=.75mm, rightrule=.15mm, toprule=.15mm, bottomrule=.15mm, breakable, opacityback=0, left=2mm, colframe=quarto-callout-tip-color-frame]
\begin{minipage}[t]{5.5mm}
\textcolor{quarto-callout-tip-color}{\faLightbulb}
\end{minipage}%
\begin{minipage}[t]{\textwidth - 5.5mm}

\textbf{Solution}\vspace{2mm}

Will be added after section

\end{minipage}%
\end{tcolorbox}

\hypertarget{section-problem-1}{%
\subsection*{✏️ Section Problem 1}\label{section-problem-1}}
\addcontentsline{toc}{subsection}{✏️ Section Problem 1}

Show that \(P\) is subadditive, i.e.~if \(\{A_i\}^{\infty}_{i=1}\) is
\textit{any} collection of sets in \(\mathcal{F}\), then
\[P\left( \bigcup_{i=1}^{\infty} A_i\right) \leq  \sum_{i=1}^{\infty} P(A_i)\]

\begin{tcolorbox}[enhanced jigsaw, arc=.35mm, colback=white, leftrule=.75mm, rightrule=.15mm, toprule=.15mm, bottomrule=.15mm, breakable, opacityback=0, left=2mm, colframe=quarto-callout-tip-color-frame]
\begin{minipage}[t]{5.5mm}
\textcolor{quarto-callout-tip-color}{\faLightbulb}
\end{minipage}%
\begin{minipage}[t]{\textwidth - 5.5mm}

\textbf{Solution}\vspace{2mm}

Will be added after section

\end{minipage}%
\end{tcolorbox}

\hypertarget{section-problem-2}{%
\subsection*{✏️ Section Problem 2}\label{section-problem-2}}
\addcontentsline{toc}{subsection}{✏️ Section Problem 2}

Let \(A_1,A_2,...\) be a sequence of sets in \(\mathcal{F}\) such that
\(A_n\uparrow A\) as \(n\to \infty\). Show that,
\(P(A_n) \xrightarrow{n \to \infty} P(A)\). Show that the result also
holds for \(A_n \downarrow A\). Does the result hold for any measure?

\begin{tcolorbox}[enhanced jigsaw, arc=.35mm, colback=white, leftrule=.75mm, rightrule=.15mm, toprule=.15mm, bottomrule=.15mm, breakable, opacityback=0, left=2mm, colframe=quarto-callout-tip-color-frame]
\begin{minipage}[t]{5.5mm}
\textcolor{quarto-callout-tip-color}{\faLightbulb}
\end{minipage}%
\begin{minipage}[t]{\textwidth - 5.5mm}

\textbf{Solution}\vspace{2mm}

Will be added after section

\end{minipage}%
\end{tcolorbox}

\hypertarget{random-variables-and-random-vectors}{%
\section*{Random Variables and Random
Vectors}\label{random-variables-and-random-vectors}}
\addcontentsline{toc}{section}{Random Variables and Random Vectors}

\markright{Random Variables and Random Vectors}

\emph{Definition 2.6.1 from textbook}

\leavevmode\vadjust pre{\hypertarget{def-random-variable}{}}%
\begin{definition}[]\label{def-random-variable}

A \emph{random variable} is a measurable function \(X\) from \(\sigma\)
to \(\mathbb{R}\) where ``measurable'' means that the preimage
\(X^{-1}(\mathcal{B}) = \{\omega \in \sigma : X(\omega) \in \mathcal{B}\}\)
is in \(\mathcal{F}\) for all Borel sets \(\mathcal{B}\).

\end{definition}

\leavevmode\vadjust pre{\hypertarget{def-random-vector}{}}%
\begin{definition}[]\label{def-random-vector}

A \emph{random vector} in \(\mathbb{R}^n\) is a measurable function
\(\mathbf{X}\) from \(\sigma\) into \(\mathbb{R}^n\), where
``measurable'' means that the preimage
\(\mathbf{X}^{-1} (\mathcal{B}) \equiv \{\omega \in \sigma: \mathbf{X}(\omega) \in \mathcal{B}\}\)
is in \(\mathcal{F}\) for all Borel sets \(\mathcal{B}\) in
\(\mathbb{R}^n\).

\end{definition}

\hypertarget{pencil-2.6.2}{%
\subsection*{✏️ Pencil 2.6.2}\label{pencil-2.6.2}}
\addcontentsline{toc}{subsection}{✏️ Pencil 2.6.2}

Show that if \(X\) is a random variable, then so is \(g(X)\) for any
measurable function \(g\).

\begin{tcolorbox}[enhanced jigsaw, arc=.35mm, colback=white, leftrule=.75mm, rightrule=.15mm, toprule=.15mm, bottomrule=.15mm, breakable, opacityback=0, left=2mm, colframe=quarto-callout-tip-color-frame]
\begin{minipage}[t]{5.5mm}
\textcolor{quarto-callout-tip-color}{\faLightbulb}
\end{minipage}%
\begin{minipage}[t]{\textwidth - 5.5mm}

\textbf{Solution}\vspace{2mm}

Will be added after section

\end{minipage}%
\end{tcolorbox}

\hypertarget{pencil-2.7.2}{%
\subsection*{✏️ Pencil 2.7.2}\label{pencil-2.7.2}}
\addcontentsline{toc}{subsection}{✏️ Pencil 2.7.2}

Let \(F(x,y)\) be a bivariate CDF. Prove that for all \(a_1 \leq b_1\)
and \(a_2 \leq b_2\),
\[F(b_1, b_2)-F(a_1, b_2)-F(b_1,a_2)+F(a_1,a_2) \geq 0\]

\begin{tcolorbox}[enhanced jigsaw, arc=.35mm, colback=white, leftrule=.75mm, rightrule=.15mm, toprule=.15mm, bottomrule=.15mm, breakable, opacityback=0, left=2mm, colframe=quarto-callout-tip-color-frame]
\begin{minipage}[t]{5.5mm}
\textcolor{quarto-callout-tip-color}{\faLightbulb}
\end{minipage}%
\begin{minipage}[t]{\textwidth - 5.5mm}

\textbf{Solution}\vspace{2mm}

Will be added after section

\end{minipage}%
\end{tcolorbox}

\hypertarget{limits-of-events}{%
\section*{Limits of Events}\label{limits-of-events}}
\addcontentsline{toc}{section}{Limits of Events}

\markright{Limits of Events}

\emph{Section 2.8 from textbook}

\leavevmode\vadjust pre{\hypertarget{def-lim}{}}%
\begin{definition}[]\label{def-lim}

Let \(A_1, A_2,...\) be a sequence of events on
\((\Omega, \mathcal{F}, P)\). Define,

\[B_n = \bigcup_{m=n}^{\infty} A_m ,\hspace{0.2cm} C_n = \bigcap_{m=n}^{\infty} A_m, \hspace{0.2cm} n\geq 1 \]

Then, \(C_n \subseteq A_n \subseteq B_n\), \$\forall n\geq 1 \$, and
\(\{B_n\}^{\infty}_{n=1}\) and \(\{C_n\}^{\infty}_{n=1}\) are decreasing
and increasing respectively. Let \(B\) and \(C\) be their respective
limiting sets. Then,

\[B = \bigcap_{n=1}^{\infty}\bigcup_{m=n}^{\infty} A_m ,\hspace{0.2cm} C = \bigcup_{n=1}^{\infty} \bigcap_{m=n}^{\infty} A_m  \]
We call \(B\) and \(C\), \(\limsup\limits_{n \to \infty}A_n\) and
\(\liminf \limits_{n \to \infty}A_n\) respectively.

\end{definition}

\hypertarget{section-problem-3}{%
\subsection*{✏️ Section Problem 3}\label{section-problem-3}}
\addcontentsline{toc}{subsection}{✏️ Section Problem 3}

Show that
\(B = \{\omega \in \Omega : \omega \in A_n \text{ for infinitely many values of } n\}\).

\begin{tcolorbox}[enhanced jigsaw, arc=.35mm, colback=white, leftrule=.75mm, rightrule=.15mm, toprule=.15mm, bottomrule=.15mm, breakable, opacityback=0, left=2mm, colframe=quarto-callout-tip-color-frame]
\begin{minipage}[t]{5.5mm}
\textcolor{quarto-callout-tip-color}{\faLightbulb}
\end{minipage}%
\begin{minipage}[t]{\textwidth - 5.5mm}

\textbf{Solution}\vspace{2mm}

Will be added after section.

\end{minipage}%
\end{tcolorbox}

\hypertarget{section-problem-4}{%
\subsection*{✏️ Section Problem 4}\label{section-problem-4}}
\addcontentsline{toc}{subsection}{✏️ Section Problem 4}

Show that
\(C = \{\omega \in \Omega: \omega \in A_n \text{ for all but finitely many values of } n\}\).

\begin{tcolorbox}[enhanced jigsaw, arc=.35mm, colback=white, leftrule=.75mm, rightrule=.15mm, toprule=.15mm, bottomrule=.15mm, breakable, opacityback=0, left=2mm, colframe=quarto-callout-tip-color-frame]
\begin{minipage}[t]{5.5mm}
\textcolor{quarto-callout-tip-color}{\faLightbulb}
\end{minipage}%
\begin{minipage}[t]{\textwidth - 5.5mm}

\textbf{Solution}\vspace{2mm}

Will be added after section

\end{minipage}%
\end{tcolorbox}

\leavevmode\vadjust pre{\hypertarget{def-limits-to-An}{}}%
\begin{definition}[]\label{def-limits-to-An}

We say that the sequence of events \(\{A_n\}^{\infty}_{n=1}\) ``has a
limit \(A\)'', if \(B = C\) and the common set is equal to \(A\). In
that case, we write \(\lim\limits_{n\to \infty} A_n = A\).

\end{definition}

\hypertarget{section-problem-5}{%
\subsection*{✏️ Section Problem 5}\label{section-problem-5}}
\addcontentsline{toc}{subsection}{✏️ Section Problem 5}

Assume above is the case. Show that \(A \in \mathcal{F}.\)

\begin{tcolorbox}[enhanced jigsaw, arc=.35mm, colback=white, leftrule=.75mm, rightrule=.15mm, toprule=.15mm, bottomrule=.15mm, breakable, opacityback=0, left=2mm, colframe=quarto-callout-tip-color-frame]
\begin{minipage}[t]{5.5mm}
\textcolor{quarto-callout-tip-color}{\faLightbulb}
\end{minipage}%
\begin{minipage}[t]{\textwidth - 5.5mm}

\textbf{Solution}\vspace{2mm}

Will be added after section

\end{minipage}%
\end{tcolorbox}

\hypertarget{section-problem-6}{%
\subsection*{✏️ Section Problem 6}\label{section-problem-6}}
\addcontentsline{toc}{subsection}{✏️ Section Problem 6}

Show that \(\lim\limits_{n \to \infty} P(A_n) = P(A)\).

\begin{tcolorbox}[enhanced jigsaw, colbacktitle=quarto-callout-note-color!10!white, leftrule=.75mm, coltitle=black, rightrule=.15mm, toptitle=1mm, bottomrule=.15mm, breakable, arc=.35mm, colback=white, titlerule=0mm, bottomtitle=1mm, toprule=.15mm, title=\textcolor{quarto-callout-note-color}{\faInfo}\hspace{0.5em}{Tip: A useful inequality}, left=2mm, opacityback=0, opacitybacktitle=0.6, colframe=quarto-callout-note-color-frame]

\[P(\liminf\limits_{n \to \infty}A_n) \leq \liminf\limits_{n \to \infty}P(A_n) \leq \limsup\limits_{n \to \infty}P(A_n) \leq P(\limsup\limits_{n \to \infty}A_n)\]

\end{tcolorbox}

\begin{tcolorbox}[enhanced jigsaw, arc=.35mm, colback=white, leftrule=.75mm, rightrule=.15mm, toprule=.15mm, bottomrule=.15mm, breakable, opacityback=0, left=2mm, colframe=quarto-callout-tip-color-frame]
\begin{minipage}[t]{5.5mm}
\textcolor{quarto-callout-tip-color}{\faLightbulb}
\end{minipage}%
\begin{minipage}[t]{\textwidth - 5.5mm}

\textbf{Solution}\vspace{2mm}

Will be added after section

\end{minipage}%
\end{tcolorbox}

\hypertarget{next-week}{%
\section*{Next Week}\label{next-week}}
\addcontentsline{toc}{section}{Next Week}

\markright{Next Week}

Next week, we will discuss:

\begin{itemize}
\tightlist
\item
  Joe's favourtie theorem: \(\pi-\lambda\) theorem
\item
  Independence of random variables
\item
  A primer on reasoning by representation
\end{itemize}

\(\,\)



\end{document}
